\documentclass{article}
\usepackage{listings}
\usepackage{geometry}
\usepackage{graphicx}
\usepackage{amssymb}
\usepackage{amsmath}
\usepackage[english]{babel}
\usepackage{amsthm}
\usepackage{mathrsfs}

\newtheorem{theorem}{Theorem}[section]
\newtheorem{corollary}{Corollary}[theorem]
\newtheorem{lemma}[theorem]{Lemma}
\newtheorem{definition}{Definition}[section]
\newtheorem{property}{Property}[section]
\newtheorem{proposition}{Proposition}[section]
\newtheorem{example}{Example}[section]

\geometry{legalpaper, margin=3cm}
\setlength{\parindent}{0pt}


\begin{document}

\title{PSFPN - Study of a Gröbner basis of two variables}

\author{Auguste WARME-JANVILLE, Adam PULVAL-DADY}

\maketitle

\section{Ideals}

\begin{definition}[Ideal]
    Let $(\mathbb{K}, +, \cdot)$ be a ring. $I$ is an ideal of $\mathbb{K}$ if : 
    \begin{itemize}
        \item $I$ is a subgroup of $(\mathbb{K}, +)$
        \item For all $x \in \mathbb{K}$ and $i \in I$, $x \cdot i \in I$
    \end{itemize}
\end{definition}

We will assume that all the rings we work on are commutative. 

\begin{definition}[Principal Ideal]
    An ideal $I$ of a ring $\mathbb{K}$ is called principal if it is generated by a single element $f$ of $\mathbb{K}$, \textit{i.e.}
    \begin{displaymath}
        I = \{f \cdot a | \forall a \in \mathbb{K}\}
    \end{displaymath}
    Such an ideal will be denoted $\langle a \rangle$.
\end{definition}

\section{Monomial orderings}

\begin{definition}[Monomial ordering]
    A monomial ordering $\prec$ on $\mathbb{K}[x, y]$ is any order satisfying : 
    \begin{itemize}
        \item $\prec$ is a total order
        \item If $w \neq 1$ is a monomial of $\mathbb{K}[x, y]$ then $1 \prec w$
        \item If $w$ is a monomial of $\mathbb{K}[x, y]$ and $u \prec v$ then $uw \prec vw$
    \end{itemize}
\end{definition}

\begin{definition}[Lexicographic order]
    The lexicographic order $\prec_{lex}$ is a monomial order over $\mathbb{K}[x, y]$ defined as :
    \begin{itemize}
        \item For all $i \in \mathbb{N}$, $y^{i} \prec_{lex} x$
    \end{itemize}
\end{definition}

\begin{proposition}
    For all $i, j, k, l \in \mathbb{N}$, $x^{i}y^{j} \prec_{lex} x^{k}y^{l}$ if and only if one of the following conditions are true : 
    \begin{itemize}
        \item $i < k$
        \item $i = k$ and $j < l$
    \end{itemize}
\end{proposition}

\begin{proof}
    
\end{proof}

\begin{definition}[drl order]
    The drl order $\prec_{drl}$ is a monomial order over $\mathbb{K}[x, y]$ defined as : 
    \begin{itemize}
        \item $y \prec_{drl} x$
        \item For all $i, j, k, l \in \mathbb{N}$, $x^{i}y^{j} \prec_{drl} x^{k}y^{l}$ if and only if one of the following conditions are true : 
        \begin{itemize}
            \item $i + j < k + l$
            \item $i + j = k + l$ and $j > l$
        \end{itemize}
    \end{itemize}
\end{definition}

\begin{example}
    The first monomials for drl order appearing in $\mathbb{K}[x, y]$ are : \\
    $1 \
    {drl} y \prec_{drl} x \prec_{drl} y^{2} \prec_{drl} xy \prec_{drl} x^{2} \prec_{drl} y^{3} \prec_{drl} xy^{2} \prec_{drl} x^{2} y \prec_{drl} x^{3} \prec_{drl} ...$
\end{example}

\begin{definition}
    Let $f \in \mathbb{K}[x, y]$ be a polynomial and $\prec$ a monomial order. 
    \begin{itemize}
        \item $LM_{\prec}(f)$ is the leading monomial of $f$ for $\prec$
        \item $LC_{\prec}(f)$ is the leading coefficient of $f$ for $\prec$
        \item $LT_{\prec}(f) = LM_{\prec}(f) \times LC_{\prec}(f)$ is the leading term of $f$ for $\prec$
    \end{itemize}
\end{definition}

\begin{lemma}
    Let $F$ and $G$ be two polynomials of $\mathbb{K}[x, y]$. Then, 
    \begin{displaymath}
        LM_{\prec}(FG) = LM_{\prec}(F)LM_{\prec}(G)
    \end{displaymath}
\end{lemma}

\begin{proof}
    
\end{proof}

\section{Gröbner Basis}

\begin{definition}[Gröbner basis]
    Let $I$ be an ideal of $\mathbb{K}[x, y]$. $\mathscr{G} \subseteq I$ is a Gröbner basis of $I$ for the monomial order $\prec$ if for all $f \in I$, there exists $g \in \mathscr{G}$ such that $LM_{\prec}(g)$ divides $LM_{\prec}(f)$. 
\end{definition}

\begin{definition}[Staircase]
    Let $I$ be an ideal of $\mathbb{K}[x, y]$. The staircase $E(I)$ for the monomial order $\prec$ is the set of all the monomials that are not leading monomials of $I$ for $\prec$. It can be expressed as follows : 
    \begin{displaymath}
        E(I) = \{ x^{i}y^{j} | \forall (i, j) \in \mathbb{N}^{2}, \nexists f \in I, LM(f) = x^{i}y^{j}\}
    \end{displaymath}
\end{definition}

\begin{proposition}
    Let $I$ be an ideal of $\mathbb{K}[x, y]$. Let $\mathscr{G} \subseteq I$ be a Gröbner basis of $I$ for the monomial order $\prec$. Let $m \in I$ be a monomial. $m \in E(I)$ if for all $g \in \mathscr{G}$, $LM_{\prec}(g)$ does not divides $m$.
\end{proposition}

\begin{proof}
    
\end{proof}

\begin{lemma}
    Let $I$ be an ideal of $\mathbb{K}[x, y]$. The staircase of $I$ if finite if and only if there exists $k, l \in \mathbb{N}$ such that :
    \begin{itemize}
        \item $x^{k}$ is a leading monomial of I
        \item $y^{l}$ is a leading monomial of I
    \end{itemize}
\end{lemma}

\begin{proof}
    
\end{proof}

\begin{theorem}
    Let $I$ be an ideal of $\mathbb{K}[x, y]$. If the staircase of $I$ for the lexicographic order is finite, there exists a non zero polynomial with monomials purely in $y$ in $I$.
\end{theorem}

\begin{proof}
    
\end{proof}

\begin{proposition}
    Let $I$ be an ideal of $\mathbb{K}[x, y]$. If $I$ is principal and $I \neq \langle 1 \rangle$, the staircase of $I$ if not finite. 
\end{proposition}

\begin{proof}
    
\end{proof}

\end{document}