\documentclass{article}
\usepackage{listings}
\usepackage{geometry}
\usepackage{graphicx}
\usepackage{amssymb}
\usepackage{amsmath}
\usepackage{hyperref}
\usepackage[english]{babel}
\usepackage{amsthm}
\usepackage{mathrsfs}
\usepackage{blindtext}
\usepackage{titlesec}
\usepackage{algorithm}
\usepackage{algpseudocode}

\newtheorem{theorem}{Theorem}[section]
\newtheorem{corollary}{Corollary}[theorem]
\newtheorem{lemma}[theorem]{Lemma}
\newtheorem{definition}{Definition}[section]
\newtheorem{property}{Property}[section]
\newtheorem{proposition}{Proposition}[section]
\newtheorem{example}{Example}[section]

%\geometry{legalpaper, margin=4cm}
\setlength{\parindent}{0pt}


\begin{document}

\title{PSFPN - Study of a Gröbner basis of two variables}

\author{Auguste WARME-JANVILLE, Alan PULVAL-DADY}

\maketitle

\section{Ideals}

\begin{definition}[Ideal]
    Let $(\mathbb{K}, +, \cdot)$ be a ring. $I$ is an ideal of $\mathbb{K}$ if : 
    \begin{itemize}
        \item $I$ is a subgroup of $(\mathbb{K}, +)$
        \item For all $x \in \mathbb{K}$ and $i \in I$, $x \cdot i \in I$
    \end{itemize}
\end{definition}

We will assume that all the rings we work on are commutative. 

\begin{definition}[Principal Ideal]
    An ideal $I$ of a ring $\mathbb{K}$ is called principal if it is generated by a single element $f$ of $\mathbb{K}$, \textit{i.e.}
    \begin{displaymath}
        I = \{f \cdot a \mid \forall a \in \mathbb{K}\}
    \end{displaymath}
    Such an ideal will be denoted $\langle f \rangle$.
\end{definition}

We will also define the intersection and the sum of two ideals, as they will be used later on.

\begin{definition}
    Let $I_{1}, I_{2}$ two ideals of $\mathbb{K}[x, y]$. 
    \begin{displaymath}
        I_{1} \cap I_{2} = \langle f \mid f \in I_{1} \land f \in I_{2} \rangle
    \end{displaymath}
\end{definition}

\begin{definition}
    Let $I_{1}, I_{2}$ two ideals of $\mathbb{K}[x, y]$. 
    \begin{displaymath}
        I_{1} + I_{2} = \langle f_{1} \mid f_{1} \in I_{1}, f_{2} \mid f_{2} \in I_{2} \rangle
    \end{displaymath}
\end{definition}

\section{Monomial orderings}

\begin{definition}[Monomial ordering]\label{def:monomial-order}
    A monomial ordering $\prec$ on $\mathbb{K}[x, y]$ is any order satisfying : 
    \begin{itemize}
        \item $\prec$ is a total order
        \item If $w \neq 1$ is a monomial of $\mathbb{K}[x, y]$ then $1 \prec w$
        \item If $w$ is a monomial of $\mathbb{K}[x, y]$ and if $u \prec v$ then $uw \prec vw$
    \end{itemize}
\end{definition}

\begin{definition}[Lexicographic order]\label{def:lexicographic-order}
    The lexicographic order $\prec_{lex}$ is a monomial order over $\mathbb{K}[x, y]$ defined as :
    \begin{itemize}
        \item For all $i \in \mathbb{N}$, $y^{i} \prec_{lex} x$
        \item For all $i, k \in \mathbb{N}$ such that $i<k$,
        $x^{i} \prec_{lex} x^{k}$
    \end{itemize}
\end{definition}

\begin{proposition}
    For all $i, j, k, l \in \mathbb{N}$, $x^{i}y^{j} \prec_{lex} x^{k}y^{l}$ if and only if one of the following conditions are satisfied : 
    \begin{itemize}
        \item $i < k$
        \item $i = k$ and $j < l$
    \end{itemize}
\end{proposition}

\begin{proof}
    $\Rightarrow$ \newline
    $y^{m} \prec_{lex} x^{m}$ \ref{def:lexicographic-order} \newline
    $y^{m+l} \prec_{lex} y^{l}x^{m}$ \ref{def:monomial-order} \newline
    $x^{i}y^{m+l} \prec_{lex} y^{l}x^{m+i}$ \ref{def:lexicographic-order} \newline
    $j = m+l$ $ k = m + i$ and we have that $m+i>i \Rightarrow k > i$ \newline
    $\Leftarrow$ Let suppose that we have $i<k$ We can then write 
    \newline
    $x^{i}\prec_{lex} x^{k}$ \ref{def:lexicographic-order} \newline
    $x^{i}y^{j}\prec_{lex} 
    x^{k}y^{j}$ \ref{def:monomial-order}
    \newline
    $x^{k}y^{j} \prec_{lex} x^{k+1}y^{j} $ \ref{def:lexicographic-order}, \ref{def:monomial-order}
    \newline
    $x^{k}y^{j}y^{w} \prec_{lex} x^{k+1}y^{j}y^{w} $ \ref{def:monomial-order}
    \newline
    $x^{i}y^{j}\prec_{lex} 
    x^{k}y^{j}\prec_{lex} x^{k'}y^{l} $ \ref{def:monomial-order}
    \newline
    $x^{i}y^{j}\prec_{lex} x^{k'}y^{l} $
    
    
\end{proof}

\begin{definition}[DRL order]
    The DRL order $\prec_{drl}$ is a monomial order over $\mathbb{K}[x, y]$ defined as : 
    \begin{itemize}
        \item $y \prec_{drl} x$
        \item For all $i, j, k, l \in \mathbb{N}$, $x^{i}y^{j} \prec_{drl} x^{k}y^{l}$ if and only if one of the following conditions are satisfied : 
        \begin{itemize}
            \item $i + j < k + l$
            \item $i + j = k + l$ and $j > l$
        \end{itemize}
    \end{itemize}
\end{definition}

\begin{example}
    The first monomials for the DRL order appearing in $\mathbb{K}[x, y]$ are : \\
    $1 \prec_{drl} y \prec_{drl} x \prec_{drl} y^{2} \prec_{drl} xy \prec_{drl} x^{2} \prec_{drl} y^{3} \prec_{drl} xy^{2} \prec_{drl} x^{2} y \prec_{drl} x^{3} \prec_{drl} ...$
\end{example}

\begin{definition} 
    Let $f \in \mathbb{K}[x, y]$ be a polynomial and $\prec$ a monomial order. 
    \begin{itemize}
        \item $LM_{\prec}(f)$ is the leading monomial of $f$ for $\prec$
        \item $LC_{\prec}(f)$ is the leading coefficient of $f$ for $\prec$
        \item $LT_{\prec}(f) = LM_{\prec}(f) \cdot LC_{\prec}(f)$ is the leading term of $f$ for $\prec$
    \end{itemize}
\end{definition}

\begin{lemma} \label{lemma:lm-mult}
    Let $f$ and $g$ be two polynomials of $\mathbb{K}[x, y]$, and $\prec$ a monomial ordering. Then, 
    \begin{displaymath}
        LM_{\prec}(f \cdot g) = LM_{\prec}(f) \cdot LM_{\prec}(g)
    \end{displaymath}
\end{lemma}

\begin{proof}
    Let us first rewrite the polynomials $F$ and $G$. We have $f = LM(f) + f^{\prime}$ and $g = LM(g) + g^{\prime}$, with $f^{\prime}, g^{\prime}$ in $\mathbb{K}[x, y]$. Then, we can multiply them and get :
    \begin{displaymath}
        f \cdot g = LM(f)LM(g) + LM(f)g^{\prime} + LM(g)f^{\prime} + f^{\prime}g^{\prime}
    \end{displaymath}
    Since $LM_{\prec}(g^{\prime}) \prec LM_{\prec}(g)$ and $LM_{\prec}(f^{\prime}) \prec LM_{\prec}(f)$, we get that $LM_{\prec}(f \cdot g) = LM_{\prec}(f) \cdot LM_{\prec}(g)$ 
\end{proof}

\section{Gröbner Bases}

In the section, we will denote $I$ an ideal of the ring $\mathbb{K}[x, y]$.

\begin{definition}[Gröbner basis]
    A finite subset $\mathscr{G} \subseteq I$ is said to be a Gröbner basis of $I$ for the monomial order $\prec$ if for all $f \in I$, there exists $g \in \mathscr{G}$ such that $LM_{\prec}(g)$ divides $LM_{\prec}(f)$. 
\end{definition}

\begin{proposition}
    Let $\mathscr{G} \subseteq I$ be a Gröbner basis of $I$ for the monomial order $\prec$. Then, $\mathscr{G}$ generates $I$.
\end{proposition}

\begin{proof}
    Suppose $\mathscr{G}$ does not generate $I$. So there exists $f \in I$ such that $f$ isn't an algebreaic combination of elements of $\mathscr{G}$
\end{proof}

\begin{definition}[Staircase]
    The staircase $E(I)$ for the monomial order $\prec$ is the set of all the monomials that are not leading monomials of $I$ for $\prec$. It can be expressed as follows : 
    \begin{displaymath}
        E(I) = \{ x^{i}y^{j} \mid \forall (i, j) \in \mathbb{N}^{2}, \nexists f \in I, LM(f) = x^{i}y^{j}\}
    \end{displaymath}
\end{definition}

\begin{proposition} \label{proposition:monomial-in-staircase}
    Let $\mathscr{G} \subseteq I$ be a Gröbner basis of $I$ for the monomial order $\prec$. Let $m \in I$ be a monomial. $m \in E(I)$ if for all $g \in \mathscr{G}$, $LM_{\prec}(g)$ does not divide $m$.
\end{proposition}

\begin{proof}
    Assume that there exists $g \in \mathscr{G}$ such that $LM_{\prec}(g)$ divides $m$. Then, there exists a monomial $k \in \mathbb{K}[x, y]$ such that $m = k \cdot LM_{\prec}(g)$. For instance, one could choose $k = 1$ or $k = g$. We just constructed a polynomial of $I$ with $m$ as its leading monomial. So $m \notin E(I)$, hence the result by contraposition.
\end{proof}

\begin{lemma} \label{lemma:finite-staircase-lm}
    The staircase of $I$ is finite if and only if there exists $k, l \in \mathbb{N}$ such that :
    \begin{itemize}
        \item $x^{k}$ is a leading monomial of I
        \item $y^{l}$ is a leading monomial of I
    \end{itemize}
\end{lemma}

\begin{proof} % Must doublecheck this proof 
    $\Rightarrow$ Assume the staircase of $I$ is finite, and let $N = Card(E(I))$. Suppose there exists no $k, l \in \mathbb{N}$ such that $x^{k}$ and $y^{l}$ are leading monomials of I. Let $\mathscr{G}$ be a Gröbner basis of $I$. Then all the leading monomials of $\mathscr{G}$ are of the form $x^{m}y^{n}$, $m,n > 0$. So for all $k, l > 0$, $x^{k}$ and $y^{l}$ are in $E(I)$. Hence $E(I)$ is not finite. \\ % Must prove that xk and yl exist
    $\Leftarrow$ Assume there exists $k, l \in \mathbb{N}$ such that $x^{k}$ and $y^{l}$ are leading monomials of I. Then, $x^{k}y^{l}$ is also a leading monomial of $I$ (Lemma \ref{lemma:lm-mult}), so the the size of $E(I)$ is at most equal to $kl$. Thus the staircase of $I$ is finite.
\end{proof}

\begin{proposition} \label{proposition:staircase-finite-any-order}
    If the staircase of $I$ is finite for a monomial ordering, then it is finite for any monomial ordering.
\end{proposition}

\begin{proof}
    Let $E(I)$ be the finite staircase for the order $\prec_{1}$. Then, there exists $k, l \in \mathbb{N}$ such that $x^{k_{1}}$ and $y^{l_{1}}$ are leading monomials of I (Lemma \ref{lemma:finite-staircase-lm}). $E(I)$ has size equal to $k_{1}l_{1}$ at most. Let $\prec_{2}$ be any monomial order different than $\prec_{1}$, and suppose $E(I)$ is not finite for $\prec_{2}$. Following the same idea as in the proof of Lemma \ref{lemma:finite-staircase-lm}, we can state that all the leading monomials for $\prec_{2}$ are of the form $x^{i}y^{j}$, $i, j \in \mathbb{N}$. 
\end{proof}

\begin{proposition}
    The staircase of $I$ for a monomial order $\prec$ is a basis of $\mathbb{K}[x, y] / I$ as a $\mathbb{K}$ vector space.
\end{proposition}

\begin{proof}
    All the monomials of $E(I)$ are monomials of the form $x^{i}y^{j}$, with all the $(i, j)$ different pairwise. So the monomials of $E(I)$ are linearly independent. $\mathbb{K}[x,y] / I$ is the quotient ring where for all $f \in \mathbb{K}[x,y] / I$, $\nexists m \in LM(I)$ such that $m$ divides $LM(f)$. Thus, by Proposition \ref{proposition:monomial-in-staircase}, $E(I)$ generates $\mathbb{K}[x,y] / I$.
\end{proof}

\begin{theorem}
    If the staircase for the monomial ordering $\prec_{1}$ is finite, then the staircase for the monomial ordering $\prec_{2}$ is also finite and has the same cardinal.
\end{theorem}

\begin{proof}
    Let $\prec_{1}$ a monomial order such that $E_{\prec_{1}}(I)$ is finite. As proven in Proposition \ref{proposition:staircase-finite-any-order}, $E_{\prec_{2}}(I)$ is finite. Let us show that $\# E_{\prec_{1}}(I) = \# E_{\prec_{2}}(I)$, \textit{i.e.} $dim(E_{\prec_{1}}(I)) = dim(E_{\prec_{2}}(I))$. 
\end{proof}

\begin{theorem}
    If the staircase of $I$ for the lexicographic order is finite, there exists a non zero polynomial with monomials in $y$ only in $I$.
\end{theorem}

\begin{proof}
    Let us assume the staircase of $I$ for $\prec_{lex}$ if finite. Then, there exists $l \in \mathbb{N}$ such that $y^{l}$ is a leading monomial of $I$. Let $f \in I$ be a polynomial with $LM(f) = y^{l}$. We can rewrite $f$ as $f = \alpha x^{y} + f^{\prime}$. As for all $k \in \mathbb{N}, y^{k} \prec_{lex} x$ and $LM(f) = y^{l}$, all the monomials of $f^{\prime}$ must contain only $y$ as a variable. Thus,  $f \in I$ is purely in $y$.
\end{proof}

\begin{proposition}
    If $I$ is principal and $I \neq \langle 1 \rangle$, the staircase of $I$ is not finite. 
\end{proposition}

\begin{proof}
    Let $f$ be the generator of $I$. Suppose $E(I)$ is finite. So there exists $(k, l) \in \mathbb{N}^{2}$ such that $x^{k}$ and $y^{l}$ are leading monomials of $I$. Then, there exists $(i, j) \in \mathbb{N}^{2}$ such that $i \leq k, j \leq l$ and $x^{k} = LM_{\prec}(f) \cdot x^{k - i}$ and $y^{l} = LM_{\prec}(f) \cdot y^{l - j}$. So $LM_{\prec}(f)$ would have to be either only in $x$ or in $y$, which is not possible as it is unique : contradiction. Thus, $E(I)$ is not finite. 
\end{proof}

\begin{algorithm}
\caption{Division algorithm over $\mathbb{K}[x, y]$}\label{alg:div}
\textbf{Input : }Gröbner basis G of an Ideal I for $\prec$+ \\
\textbf{Output : }Normal form of f
\begin{algorithmic}
    \State $h \gets 0$
    \While{$f \neq 0$} 
        \If{there exists g $\in$ $\mathscr{G}$ such that $LM_{\prec}(g)$ divides $LM_{\prec}(f)$}
            \State $f \gets f-\frac{LT_{\prec}(f)}{LT_{\prec}(g)}g$
        \Else
            \State $h \gets h + LT_{\prec}(f)$
            \State $f \gets f - LT_{\prec}(f)$
        \EndIf
    \EndWhile
    \State \Return $h$
\end{algorithmic}
\end{algorithm}

\begin{definition}
    Let $h$ be the output of Algorithm \ref{alg:div}. $h$ is called the normal form of $f$ with respect to $\mathscr{G}$ and $\prec$. In a more formal way, 
    \begin{displaymath}
        h = NF(f, \mathscr{G}, \prec)
    \end{displaymath}
\end{definition}

\begin{proposition}
    Let $h$ be the output of the above algorithm. Every monomial of $h$ is in the staircase of $I$.
\end{proposition}
    
\begin{proof}
    In the beginning, $h = 0$. Then, if $\nexists g \in \mathscr{G}$ such that $LM_{\prec}(g)$ divides $LM_{\prec}(f)$, $h \gets LT_{\prec}(f)$. \\
    Therefore, every monomial of $h$ is in the staircase of $I$, as proved above (Proposition \ref{proposition:monomial-in-staircase}). 
\end{proof}

\begin{theorem}
    Let $h$ be the output of the above algorithm. Then, $h = 0$ if and only if $f \in I$.
\end{theorem}

\begin{proof}
\end{proof}

\begin{corollary}
    Let $h$ be the output of the above algorithm. $h$ does not depend on the choices taken during the execution of the algorithm. 
\end{corollary}

\begin{proof}
    
\end{proof}

\section{Varieties : equation solving point of view}

\begin{definition}
    The variety of an ideal $I$ is the set as defined below : 
    \begin{displaymath}
        V(I) = \{ (a, b) \in \overline{\mathbb{K}} \mid \forall f \in I, f(a, b) = 0 \}
    \end{displaymath}
    Where $\overline{\mathbb{K}}$ stands for the algebraic closure of $\mathbb{K}$. 
\end{definition}

\begin{proposition}
    Let $I_{1}, I_{2}$ two ideals of $\mathbb{K}[x, y]$. 
    \begin{enumerate}
        \item[(i)] $V(I_{1} \cap I_{2}) = V(I_{1}) \cup V(I_{2})$
        \item[(ii)] $V(I_{1} + I_{2}) = V(I_{1}) \cap V(I_{2})$
    \end{enumerate}
\end{proposition}
    
\begin{proof}
    Let us first write the formal definitions of $V(I_{1})$ and $V(I_{2})$. We have : \\
    \begin{displaymath}
        V(I_{1}) = \{ (a, b) \in \overline{\mathbb{K}} \mid \forall f_{1} \in I_{1}, f_{1}(a, b) = 0 \}, \;
        V(I_{2}) = \{ (c, d) \in \overline{\mathbb{K}} \mid \forall f_{2} \in I_{2}, f_{2}(a, b) = 0 \} 
    \end{displaymath}
    
    \begin{enumerate}
        \item[(i)] 
        \begin{align*}
            V(I_{1} \cap I_{2})
            & = V( \langle f \mid f \in I_{1} \land f \in I_{2} \rangle) \\
            & = \{(a, b) \in \overline{\mathbb{K}} \mid \forall f \in I_{1} \cap I_{2}, f(a, b) = 0\} \\
            & = \{ (a, b) \in \overline{\mathbb{K}} \mid \forall f_{1} \in I_{1}, \forall f_{2} \in I_{2}, f_{1}f_{2}(a, b) = 0 \} \\
            & = \{ (a, b) \in \overline{\mathbb{K}} \mid \forall f_{1} \in I_{1}, \forall f_{2} \in I_{2}, f_{1}(a, b) = 0 \lor f_{2}(a, b) = 0 \} \\
            & = \{ (a, b) \in \overline{\mathbb{K}} \mid \forall f_{1} \in I_{1}, f_{1}(a, b) = 0 \} \cup 
            \{ (c, d) \in \overline{\mathbb{K}} \mid \forall f_{2} \in I_{2}, f_{2}(a, b) = 0 \} \\
            & = V(I_{1}) \cup V(I_{2}) \\
        \end{align*}
        \item[(ii)] 
        \begin{align*} 
            V(I_{1} + I_{2}) 
            & = \{ (a, b) \in \overline{\mathbb{K}} \mid \forall f \in I_{1} + I_{2}, f(a, b) = 0\} \\
            & = \{ (a, b) \in \overline{\mathbb{K}} \mid \forall f_{1} \in I_{1}, \forall f_{2} \in I_{2}, f_{1}(a, b) = 0 \land f_{2}(a, b) = 0 \} \\
            & = V(I_{1}) \cap V(I_{2}) \\
        \end{align*}
    \end{enumerate}
\end{proof}

\section{Applications}

We will now study ideals based on generators of a certain form. 
First, let us define the points $p_{i} = (x_{i}, y_{i})$ for $0 \leq i \leq d$, where $y_{i} \neq y_{j}, \forall i, j$. 
Then, we will define the following polynomials : 

\begin{displaymath}
    h(y) = \prod_{i=0}^{d-1} (y - y_{i}) 
\end{displaymath}
And the polynomial $g(y) = x$ that can be defined by interpolating the points $p_{i}$ as all the $y_{i}$ are distinct. 

We can now define the following ideal : 

\begin{displaymath}
    I = \langle h(y), x - g(y) \rangle
\end{displaymath}

\begin{proposition}
    $I$ is a Gröbner basis of I for the lexicographic ordering $\prec_{lex}$.  
\end{proposition}

\begin{proof}
    Let $a \in I$. If $a$ is purely in $y$, then $a$ is a multiple of $h(y)$ and then $LM(h(y))$ divides $LM(a)$. Otherwise, $LM(a) = x^{k}$ where $k \geq 1$, so $LM(x - g(y)) = x$ divides $LM(a)$.
\end{proof}

\subsection{Ideals Intersections}

Let us now define the ideals $I_{1}$ and $I_{2}$. 
\begin{displaymath}
    I_{1} = \langle h_{1}(y), x - g_{1}(y) \rangle
\end{displaymath}
\begin{displaymath}
    I_{2} = \langle h_{2}(y), x - g_{2}(y) \rangle
\end{displaymath}

We will assume that $\gcd(h_{1}, h_{2}) = 1$.

\begin{theorem}
    \begin{displaymath}
        I_{1} \cap I_{2} = \langle h_{1}h_{2}(y), x - g_{3}(y) \rangle
    \end{displaymath}
    Where $g_{3}(y)$ satisfies : % can be computed using the CRT
    \begin{displaymath}
    \left\{
    \begin{array}{ll}
        g_{3} \equiv g_{1} \pmod {h_{1}} \\
        g_{3} \equiv g_{2} \pmod {h_{2}} \\
    \end{array}
    \right.
    \end{displaymath}
\end{theorem}

\begin{proof}
    We will first show that $I_{1} \cap I_{2} \subset \langle h_{1}h_{2}(y), x - g_{3}(y) \rangle$.
    Assume that $f \in I_{1} \cap I_{2}$. There are a few possible cases : 
    \begin{itemize}
        \item $f$ is purely in $y$. Then, $h_{1}$ divides $f$ and $h_{2}$ divides $f$. Hence $h_{1}h_{2}$ divides $f$ so $f \in \langle h_{1}h_{2}(y), x - g_{3}(y) \rangle$
        \item $f$ is 
    \end{itemize}

    Now, let us show that $\langle h_{1}h_{2}(y), x - g_{3}(y) \rangle \subset I_{1} \cap I_{2}$.

\end{proof}


Now, let us assume that $\gcd(h_{1}, {h_2}) \neq 1$. 

\begin{algorithm}
    \caption{Intersection ($I_{1}, I_{2}$)}\label{alg:intersect-2-ideals}
    \textbf{Input : }Two Ideals $I_{1}$ and $I_{2}$ as described above \\
    \textbf{Output : }$I_{1} \cap I_{2}$
\begin{algorithmic}
    \State $h_{1}, g_{1} = I_{1}$ 
    \State $h_{2}, g_{2} = I_{2}$ 
    \State $l, g \gets lcm(h_{1}, h_{2}), gcd(h_{1}, h_{2})$
    \State $p_{1}, p_{2} \gets \frac{h_{1}}{g}, \frac{h_{2}}{g}$
    \State $q_{1}, q_{2} \gets g_{1} \pmod {p_{1}}, g_{2} \pmod {p_{2}}$
    \State $q_{3} \gets g_{1} \cdot g_{2} \pmod {g}$ \Comment{Polynomials can both be reduced modulo $g$ first}
    \State $u_{1}, v_{1} \gets gcdex_{y} (p_{1}, p_{2})$ \Comment{$u_{1}, v_{1}$ represent the cofactors here}
    \State $u_{2}, v_{2} \gets gcdex_{y} (p_{1} \cdot p_{2}, g)$ \Comment{$u_{2}, v_{2}$ represent the cofactors here}
    \State $P_{1}, P_{2} \gets u_{1}q_{2}p_{1}, v_{1}q_{1}p_{2}$ \Comment{This are the CRT solutions}
    \State $P_{12} = P_{1} + P_{2}$
    \State $P_{3}, P_{4} \gets u_{2}q_{3}p_{1}p_{2}, v_{2}P_{12}g$
    \State $P_{34} = P_{3} + P_{4}$
    \State $B^{\prime} = \langle l, gP_{12} \rangle$
    \State $P_{34} \gets NF(P_{34}, B^{\prime})$
    \State $R = \langle l, NF(gP_{12}, \langle l \rangle), P_{34} \rangle$ \\
    \Return $R$
    
    
    %\State $c \gets 1$
    %\For{$j = 1$ to $n$}
    %    \For{$i = n$ to $j + 1$}
    %        \State $\theta \gets \tan^{-1}\left(\frac{A_{i, c}}{A_{i-1, c}}\right)$
    %        \State $A \gets G(i-1, i, \theta) A G(i-1, i, \theta)^{\star}$
    %    \EndFor
    %    \State $c \gets c + 1$
    %\EndFor
\end{algorithmic}
\end{algorithm}


\end{document}