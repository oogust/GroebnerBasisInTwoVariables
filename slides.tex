\documentclass{beamer}

\usepackage{listings}
\usepackage{geometry}
\usepackage{graphicx}
\usepackage{amssymb}
\usepackage{amsmath}
\usepackage{hyperref}
\usepackage[english]{babel}
\usepackage{amsthm}
\usepackage{mathrsfs}
\usepackage{blindtext}
\usepackage{algorithm}
\usepackage{algpseudocode}
\usepackage[utf8]{inputenc}
\usepackage{fancyhdr}
\usepackage{braket}
\usepackage{mathtools}
\usepackage[backend=bibtex, style=numeric]{biblatex}

\addbibresource{bibliography.bib}


%\newtheorem{theorem}{Theorem}[section]
%\newtheorem{corollary}{Corollary}[theorem]
%\newtheorem{lemma}{Lemma}[section]
%\newtheorem{definition}{Definition}[section]
\newtheorem{proposition}{Proposition}[section]
%\newtheorem{example}{Example}[section]
%\newtheorem{property}{Property}[section]
%\newtheorem{remark}{Remark}[section]
%\newtheorem{conjecture}{Conjecture}[section]

\DeclareMathOperator{\lcm}{lcm}
\DeclareMathOperator{\divides}{div}

\AtBeginSection[]
{
    \begin{frame}
        \frametitle{Table of Contents}
        \tableofcontents[currentsection]
    \end{frame}
}

\title{Gröbner bases in two variables}
\author{Auguste Warmé-Janville, Alan Pulval-Dady}
\institute{Sorbonne Université}

\begin{document}

\frame{\titlepage}

%\begin{frame}
%    \frametitle{Table of Contents}
%    \tableofcontents
%\end{frame}


\begin{frame}{Introduction}
    Given $f_1 = 0, f_r = 0$ with $f_i \in \mathbb{K}[x,y]$, $\mathbb{K}$ a field.
    We want to compute the solution of the above system.
    The Gröbner bases of the ideal $I = \langle f_1,\dots,f_r \rangle$ has advantageous proprieties to find solutions of the system or test the belonging of a polynomial tn the ideal.
    Therefore we want to study the structure of such Gröbner bases when using the lexicographical order.
\end{frame}

\begin{frame}{Remainders}
    \begin{definition}[Lexicographic order]\label{def:lexicographic-order}
        The lexicographic order $\prec_{lex}$ is the monomial order over $\mathbb{K}[x, y]$ defined as : 
        \begin{displaymath}
            \forall i \in \mathbb{N}, y^{i} \prec_{lex} x
        \end{displaymath}
    \end{definition}
    \begin{definition}[Gröbner basis]
        A finite subset $\mathscr{G} \subseteq I$ is said to be a Gröbner basis of $I$ for the monomial order $\prec$ if for all $f \in I$, there exists $g \in \mathscr{G}$ such that $LM_{\prec}(g)$ divides $LM_{\prec}(f)$. 
    \end{definition}
    \begin{definition}[Staircase]
        The staircase $E(I)$ for the monomial order $\prec$ is the set of all the monomials that are not leading monomials of $I$ for $\prec$. It can be expressed as follows : 
        \begin{displaymath}
            E(I) = \{ x^{i}y^{j} \mid \forall (i, j) \in \mathbb{N}^{2}, \nexists f \in I, LM(f) = x^{i}y^{j}\}
        \end{displaymath}
    \end{definition}
\end{frame}

\begin{frame}{Staircase}
    Consider the ideal $I = \langle y^{5} + 1, x^2 + y - 1 \rangle$ for the lexicographic ordering. \\
    All the multiples of $y^{5} + 1$ and $x^2 + y - 1$ are not in the staircase of $I$, by definition. 
    Then, we get that : 
    \begin{displaymath}
        E(I) = \{ 1, y, y^{2}, y^{3}, y^{4}, x, xy, xy^{2}, xy^{3}, xy^{4} \}
    \end{displaymath}
    \begin{theorem} \label{th:polynomial-only-in-y}
        If the staircase of $I$ for the lexicographic order is finite, there exists a non zero polynomial with monomials in $y$ only in $I$.
    \end{theorem}

    \textbf{Hint :} $\exists l, f = \alpha y^{l} + f^{\prime} \in I$. $LM(f) = \alpha y^{l}$. As for all $k \in \mathbb{N}, y^{k} \prec_{lex} x$ and $LM(f) = y^{l}$, all the monomials of $f^{\prime}$ must contain only $y$ as a variable
\end{frame}

\begin{frame}[shrink=20]{First results}
    \begin{algorithm}[H]
        \caption{Division algorithm over $\mathbb{K}[x, y]$} \label{alg:div}
        \textbf{Input : }A polynomial $f \in \mathbb{K}[x, y]$, $\mathscr{G} \subseteq I$ \\
        \textbf{Output : }Normal form of $f$
        \begin{algorithmic}
            \State $h \gets 0$
            \While{$f \neq 0$} 
                \If{there exists g $\in$ $\mathscr{G}$ such that $LM_{\prec}(g)$ divides $LM_{\prec}(f)$}
                    \State $f \gets f- \displaystyle \frac{LT_{\prec}(f)}{LT_{\prec}(g)}g$
                \Else
                    \State $h \gets h + LT_{\prec}(f)$
                    \State $f \gets f - LT_{\prec}(f)$
                \EndIf
            \EndWhile
            \State \Return $h$
        \end{algorithmic}
        \end{algorithm}
    \begin{theorem} \label{th:ideal-membership-test}
        Let $h$ be the output of Algorithm \ref{alg:div}. If $\mathscr{G}$ is a Gröbner basis of $I$ for the order $\prec$, $h = 0$ if and only if $f \in I$.
    \end{theorem}
\end{frame}

\begin{frame}{Geometric point of view}
    \begin{definition} [Affine variety]
        The variety of an ideal $I$ is the set as defined below :
        \begin{displaymath}
            V(I) = \{ (a, b) \in \overline{\mathbb{K}} \mid \forall f \in I, f(a, b) = 0 \}
        \end{displaymath}
        Where $\overline{\mathbb{K}}$ stands for the algebraic closure of $\mathbb{K}$.
    \end{definition}
    \begin{proposition}
        Let $I_{1}, I_{2}$ be two ideals of $\mathbb{K}[x, y]$.
        \begin{enumerate}
            \item[(i)] $V(I_{1} \cdot I_{2}) = V(I_{1}) \cup V(I_{2})$
            \item[(ii)] $V(I_{1} \cap I_{2}) = V(I_{1}) \cup V(I_{2})$
        \end{enumerate}
    \end{proposition}
\end{frame}




\begin{frame}{Ideals Intersections}
    Given the points $p_{i} = (x_{i}, y_{i})$ for $0 \leq i \leq d$, where $y_{i} \neq y_{j}, \forall i, j$. 
    We define the following polynomials : 
    
    \begin{displaymath}
        h(y) = \prod_{i=0}^{d-1} (y - y_{i}) 
    \end{displaymath}
    And the polynomial $g(y) = x$ that can be defined by interpolating the points $p_{i}$ as all the $y_{i}$ are distinct.
    Let us now define the ideals $I_{1}$ and $I_{2}$.
    \begin{displaymath}
        I_{1} = \langle h_{1}(y), x - g_{1}(y) \rangle
    \end{displaymath}
    \begin{displaymath}
        I_{2} = \langle h_{2}(y), x - g_{2}(y) \rangle
    \end{displaymath}
    We will assume that $\gcd(h_{1}, h_{2}) = 1$.
\end{frame}

\begin{frame}{Simplest case}
\begin{theorem}
    Let $I_{1}$ and $I_{2}$ be two ideals as defined above, with $g_{1}$ defined by interpolating the points $p_{0,\dots,d-1}$ and $g_{2}$ defined by interpolating the points $p_{d,\dots,e-1}$
    \begin{displaymath}
        I_{1} \cap I_{2} = \langle h_{1}h_{2}(y), x - g_{3}(y) \rangle
    \end{displaymath}
    Where $g_{3}(y)$ is the polynomial defined by interpolating the points $p_{0,\dots,d-1,d,\dots,e-1}$ and satisfies :
    \begin{displaymath}
    \left\{
    \begin{array}{ll}
        g_{3} = g_{1} \pmod {h_{1}} \\
        g_{3} = g_{2} \pmod {h_{2}} \\
    \end{array}
    \right.
    \end{displaymath}
\end{theorem}
\end{frame}

\begin{frame}{Case 2}
    Now, assume $\gcd(h_{1}, {h_2}) \neq 1$.

We can now write the two ideals as :
\begin{align*}
    I_{1} = \langle h_{1}(y), x - g_{1}(y) \rangle = \langle \alpha p_{1}(y), x - g_{1}(y) \rangle \\
    I_{2} = \langle h_{2}(y), x - g_{2}(y) \rangle = \langle \alpha p_{2}(y), x - g_{2}(y) \rangle
\end{align*}

Where $\alpha = \gcd(h_{1}, h_{2})$ and $\gcd(p_{1}, p_{2}) = 1$.

We will define then define the polynomials $g_{3}(y)$
\begin{displaymath}
    \left\{
    \begin{array}{ll}
        g_{3} = g_{1} \pmod {p_{1}} \\
        g_{3} = g_{2} \pmod {p_{2}} \\
    \end{array}
    \right.
\end{displaymath}
and $f(x, y)$
\begin{align*}
    \left\{
    \begin{array}{ll}
        f = x - g_{3}           & \pmod {p_{1}p_{2}}         \\
        f = (x-g_{1})(x-g_{2})  & \pmod {\alpha}             \\
    \end{array}
    \right.
\end{align*}

\begin{theorem} \label{th:inter-2-gcd-neq-1}
    Assume $\gcd(h_{1}, h_{2}) \neq 1$.
    \begin{displaymath}
        I_{1} \cap I_{2} = \langle \lcm(h_{1}, h_{2})(y), \alpha (x - g_{3}(y)), f(x,y) \rangle
    \end{displaymath}
\end{theorem}
\end{frame}

\begin{frame}{Case 2 - Generalization}
How can we generalize this process for an intersection of N ideal instead of only 2 ?
Assume we have N ideals with the same structure which are :
\begin{align*}
    I_{1} & = \langle h_{1}(y),x-g_{1}(y) \rangle    \\
          & \hspace{8mm} \vdots                      \\
    I_{N} & = \langle h_{N}(y),x-g_{N}(y) \rangle
\end{align*}
Also, we have the assumption that: 
\begin{align*}
    gcd(h_{1},\dots,h_{N}) & \neq 1                 & \forall i \in \{1,\dots,N\} \\
    gcd(h_{i},h_{i+1})     & =gcd(h_{i+1},h_{i+2})\ & \forall i \in \{1,\dots,N-2\}
\end{align*}
 We define the polynomials $y_{1},\dots,y_{N}$ such that $y_{1} = \displaystyle \frac{h_{1}}{\gcd(h_{1},\dots,h_{N})}$,$\dots$, $y_{N} = \displaystyle \frac{h_{N}}{\gcd(h_{1},\dots,h_{N})}$.

\end{frame}

\begin{frame}{Case 2 - Generalization}
    Let $f_{1}$ be the polynomial such that :
\begin{displaymath}
    \left\{
    \begin{array}{ll}
        f_{1} = x-g_{1} \pmod {y_{1}} \\
        f_{1} = x-g_{2} \pmod {y_{2}} \\
        \hspace{8mm} \vdots \\
        f_{1} = x-g_{N} \pmod{y_{N}}\\
    \end{array}
    \right.
\end{displaymath}
    And $f_{2}$ :
\begin{displaymath}
    \left\{
    \begin{array}{ll}
        f_{2} = f_{1} \pmod {\prod_{i=1}^{N} y_{i}} \\
        f_{2} = \prod^{N}_{i=1} (x-g_{i}) \pmod {\gcd(h_{1},...,h_{N})} \\
    \end{array}
    \right.
\end{displaymath}
$f_1$ and $f_2$ can be computed using the CRT.

\begin{theorem}
    When $\gcd(h_{1},\dots,h_{N})\neq 1$ and $\gcd(h_{i},h_{i+1}) = \gcd(h_{i+1},h_{i+2}) $
    \begin{displaymath}
        \bigcap_{k = 1}^{N} I_{k} = \langle \lcm(\{h_{i}\}), \gcd(h_{1},...,h_{N})f_{1},f_{2} \rangle
    \end{displaymath}
\end{theorem}
\end{frame}


\begin{frame}{Case 3}
Let us now assume that we have :
\[\gcd(h_{1},\dots,h_{N}) \neq 1\]
But this time we don't necessarily have the following proprieties :
\[\gcd(h_{1},h_{2}) = \gcd(h_{2},h_{3}) = \dots = \gcd(h_{N-1},h_{N}) \]

Which implies that there exist $i$ such that $gcd(h_{i},h_{i+1}) \neq gcd(h_{i+1},h_{i+2})$

This mean that the polynomial $H = \prod_{i=1}^{N} h_{i}$ has at least one factor which isn't gcd($h_{1},\dots,h_{N}$) with multiplicity greater than one.

We can then write H as :
\[H=\prod_{k_{i} \in \mathbb{N}
 \setminus \{0\}, y_{i}\in \mathbb{K}} (y-y_{i})^{k_{i}}\]

\end{frame}

\begin{frame}{Case 3}
    We define the set $H_{j}$ as following :
\[H_{j} = \{\ (y-y_{i}) \ |\  k_{i} \geq j\ \}\]

We define the height of H as follows : \[height(H) = \max_{k_{i}}(\frac{H}{gcd(h_{1},\dots,h_{N})})\]

Let us define \[W_{x} = H_{x}-H_{x+1}\] and \[C_{w} = \{ (x-g_{i}) \mid w \divides h_{i}, \forall 1 \leq i \leq N \}\] \newline
\end{frame}

\begin{frame}{Case 3}
    We finally define the polynomials $\forall i \in \{1,\dots,height(H)\}, \forall w \in W_{i}$ :

Case $i = 1$

\[\{ f_{i} = \prod_{c \in C_{w} } c \pmod w \}\]

Case $i > 1$

\begin{displaymath}
    \left\{
    \begin{array}{ll}
    f_{i} = \prod_{c \in C_{w} } c \pmod w \\
    f_{i} = f_{i-1} \pmod{\prod_{w^{*} \in W_{i-1}} w^{*}}\\
    \end{array}
    \right.
\end{displaymath}

for both cases, we can find $f_{i}$ using the CRT\@.
\end{frame}


\begin{frame}{Algorithm case 2}
    \begin{algorithm}[H]
    \caption{Intersect1($I_{1},\dots,I_{N}$)}\label{alg:intersect-n-ideals-equal-gcd}
    \textbf{Input : } N Ideals $I_{1},\dots,I_{N}$ as described above \\
    \textbf{Output : }$I_{1} \cap \dots\cap I_ {N}$
\begin{algorithmic}
    \State $H \coloneqq \{h_{1},..,h_{N}\}$
    \State $P1 \coloneqq LCM(H)$
    \State $h^{*} \coloneqq GCD(H)$
    \State $Y\coloneqq \{y_{1}= \frac{h_{1}}{h^{*}},\dots,y_{N}= \frac{h_{N}}{h^{*}}\}$
    \State $X \coloneqq \{x_{1} = \frac{x-g_{1}}{y_{1}},\dots,x_{N} =\frac{x-g_{N}}{y_{N}}\}$
    \State $x^{*} \coloneqq \prod^{N}_{i=1} \frac{x-g_{i}}{h^{*}}$
    \State $c\coloneqq CRT(X,Y)$
    \State $P2 \coloneqq c[1]*h^{*}$ \Comment{The CRT algorithm implemented return 2 elements the first one is the found polynomial and the second one the multiplication of each divisor $\prod^{N}_{i=1} y_{i}$}
    \State $P3 \coloneqq CRT(\{c[1],x^{*}\},\{c[2], h^{*}\})$
    \State \Return P1,P2,P3
\end{algorithmic}
\end{algorithm}
\end{frame}

\begin{frame}{Algorithm case 3}
    \begin{algorithm}[H]    \tiny
    \caption{Intersect2 ($I_{1},\dots,I_{N}$)}\label{alg:intersect-2-ideals-diff-gcd}
    \textbf{Input : } N Ideals $I_{1},\dots,I_{N}$ as described above \\
    \textbf{Output : }$I_{1} \cap \dots\cap I_ {N}$
\begin{algorithmic}
    \State $H \coloneqq \{h_{1},..,h_{N}\}$
    \State $P \coloneqq []$
    \State $C \coloneqq $ Correspondences($h_{i},g_{i}$) \Comment{List acting as the dictionary defined above}
    \State $P[1] \coloneqq LCM(H)$
    \State $H \coloneqq \prod^{N}_{i=1} h_{i}$
    \State $F \coloneqq $ListOfFactorsAndMultiplicities(H)
    \State $F \coloneqq $DecreaseAllMultiplicitiesByOne(F)
    \If{only one factor has a multiplicity greater than 0}
        \State \Return Intersect1($I_{1},\dots,I_{N}$)
    \EndIf
    \State $Last\_used\_poly\_y \coloneqq 1$
    \State $Last\_found\_poly\_x \coloneqq 1$
    \State $Nb\_It \coloneqq MaxMultiplicity(F) $ \Comment{MaxMultiplicity act like the function height() defined above}
    \For{ j from 1 to Nb\_It}

        \State $y^{*}\coloneqq \prod_{f \in H_{j}} f$ \Comment{All the factor of H with multiplicities greater or equal to j }
        \State $p_y\coloneqq \frac{\frac{P[1]}{y^{*}}}{last\_used\_poly\_y}$ \Comment{Equivalent of $W_{j}$ but instead of a set of factor, all the factor are multiplied}
        \State $X, Y\coloneqq$ BuildCRTParameters($p_y, C$)
        \State $res \coloneqq $ CRT($X,Y$)
        \If{$i > 1$}
            \State $res \coloneqq $ CRT($[res[1],last\_found\_poly\_x],[res[2],last\_used\_poly\_y]$)
        \EndIf
        \State $P[i+1] \coloneqq $ NormalForm($res[1]*y^{*},P$)
        \State $F \coloneqq $ DecreaseAllMultiplicitiesByOne(F)
    \EndFor
    \State $X,Y$ := BuildCRTParameters($GCD(h_{1},\dots,h_{N}),C$)
    \State res := CRT($X,Y\cup\{last\_used\_poly\_y\}$
    \State P[Nb\_It+2] $\coloneqq $NormalForm($res[1],P$)
    \State \Return P

\end{algorithmic}
\end{algorithm}

\end{frame}

\begin{frame}{References}
    \nocite{*}
    \printbibliography
\end{frame}

\end{document}