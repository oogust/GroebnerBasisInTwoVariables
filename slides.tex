\documentclass{beamer}

\usepackage{listings}
\usepackage{geometry}
\usepackage{graphicx}
\usepackage{amssymb}
\usepackage{amsmath}
\usepackage{hyperref}
\usepackage[english]{babel}
\usepackage{amsthm}
\usepackage{mathrsfs}
\usepackage{blindtext}
\usepackage{algorithm}
\usepackage{algpseudocode}
\usepackage[utf8]{inputenc}
\usepackage{fancyhdr}
\usepackage{braket}
\usepackage{mathtools}
\usepackage[backend=bibtex, style=numeric]{biblatex}

\addbibresource{bibliography.bib}


%\newtheorem{theorem}{Theorem}[section]
%\newtheorem{corollary}{Corollary}[theorem]
%\newtheorem{lemma}{Lemma}[section]
%\newtheorem{definition}{Definition}[section]
\newtheorem{proposition}{Proposition}[section]
%\newtheorem{example}{Example}[section]
%\newtheorem{property}{Property}[section]
%\newtheorem{remark}{Remark}[section]
%\newtheorem{conjecture}{Conjecture}[section]

\DeclareMathOperator{\lcm}{lcm}

\AtBeginSection[]
{
    \begin{frame}
        \frametitle{Table of Contents}
        \tableofcontents[currentsection]
    \end{frame}
}

\title{Gröbner bases in two variables}
\author{Auguste Warmé-Janville, Alan Pulval-Dady}
\institute{Sorbonne Université}

\begin{document}

\frame{\titlepage}

\begin{frame}
    \frametitle{Table of Contents}
    \tableofcontents
\end{frame}


\begin{frame}{Introduction}
\end{frame}

\begin{frame}{Geometric point of view}
    \begin{definition} [Affine variety]
        The variety of an ideal $I$ is the set as defined below :
        \begin{displaymath}
            V(I) = \{ (a, b) \in \overline{\mathbb{K}} \mid \forall f \in I, f(a, b) = 0 \}
        \end{displaymath}
        Where $\overline{\mathbb{K}}$ stands for the algebraic closure of $\mathbb{K}$.
    \end{definition}
    \begin{proposition}
        Let $I_{1}, I_{2}$ two ideals of $\mathbb{K}[x, y]$.
        \begin{enumerate}
            \item[(i)] $V(I_{1} \cdot I_{2}) = V(I_{1}) \cup V(I_{2})$
            \item[(ii)] $V(I_{1} \cap I_{2}) = V(I_{1}) \cup V(I_{2})$
            \item[(iii)] $V(I_{1} + I_{2}) = V(I_{1}) \cap V(I_{2})$
        \end{enumerate}
    \end{proposition}
\end{frame}

\begin{frame}{Ideals Intersections case 1}
    Let us now define the ideals $I_{1}$ and $I_{2}$.
\begin{displaymath}
    I_{1} = \langle h_{1}(y), x - g_{1}(y) \rangle
\end{displaymath}
\begin{displaymath}
    I_{2} = \langle h_{2}(y), x - g_{2}(y) \rangle
\end{displaymath}
We will assume that $\gcd(h_{1}, h_{2}) = 1$.

\begin{theorem}
    Let $I_{1}$ and $I_{2}$ be two ideals as defined above, with $g_{1}$ defined by interpolating the points $p_{0,\dots,d-1}$ and $g_{2}$ defined by interpolating the points $p_{d,\dots,e-1}$
    \begin{displaymath}
        I_{1} \cap I_{2} = \langle h_{1}h_{2}(y), x - g_{3}(y) \rangle
    \end{displaymath}
    Where $g_{3}(y)$ is the polynomial defined by interpolating the points $p_{0,\dots,d-1,d,\dots,e-1}$ and satisfies :
    \begin{displaymath}
    \left\{
    \begin{array}{ll}
        g_{3} = g_{1} \pmod {h_{1}} \\
        g_{3} = g_{2} \pmod {h_{2}} \\
    \end{array}
    \right.
    \end{displaymath}
\end{theorem}
\end{frame}

\begin{frame}{Case 2}
    Now, assume $\gcd(h_{1}, {h_2}) \neq 1$.

We can now write the two ideals as :
\begin{align*}
    I_{1} = \langle h_{1}(y), x - g_{1}(y) \rangle = \langle \alpha p_{1}(y), x - g_{1}(y) \rangle \\
    I_{2} = \langle h_{2}(y), x - g_{2}(y) \rangle = \langle \alpha p_{2}(y), x - g_{2}(y) \rangle
\end{align*}

Where $\alpha = \gcd(h_{1}, h_{2})$ and $\gcd(p_{1}, p_{2}) = 1$.

We will define then define polynomials $g_{3}(y)$
\begin{displaymath}
    \left\{
    \begin{array}{ll}
        g_{3} = g_{1} \pmod {p_{1}} \\
        g_{3} = g_{2} \pmod {p_{2}} \\
    \end{array}
    \right.
\end{displaymath}
and $f(x, y)$
\begin{align*}
    \left\{
    \begin{array}{ll}
        f = x - g_{3}           & \pmod {p_{1}p_{2}}         \\
        f = (x-g_{1})(x-g_{2})  & \pmod {\alpha}             \\
    \end{array}
    \right.
\end{align*}

\begin{theorem} \label{th:inter-2-gcd-neq-1}
    Assume $\gcd(h_{1}, h_{2}) \neq 1$.
    \begin{displaymath}
        I_{1} \cap I_{2} = \langle \lcm(h_{1}, h_{2})(y), \alpha (x - g_{3}(y)), f(x,y) \rangle
    \end{displaymath}
\end{theorem}
\end{frame}

\begin{frame}{Case 2 - Generalization}
How can we generalize this process for an intersection of N ideal instead of only 2 ?
Let assume we have N Ideal with the same structure which are :

$I_{1} = \langle h_{1}(y),x-g_{1}(y) \rangle, I_{2} = \langle h_{2}(y),x-g_{2}(y) \rangle, \dots ,I_{N} = \langle h_{N}(y),x-g_{N}(y) \rangle$


Also, we have the assumption that: \[gcd(h_{1},\dots,h_{N})\neq1\ \forall i \in \{1,\dots,N\}\]
and \[gcd(h_{i},h_{i+1})=gcd(h_{i+1},h_{i+2})\ \forall i \in \{1,\dots,N-2\}\]
 We define the polynomials $y_{1},\dots,y_{N}$ such that $y_{1} = \displaystyle \frac{h_{1}}{\gcd(h_{1},\dots,h_{N})}$,$\dots$, $y_{N} = \displaystyle \frac{h_{N}}{\gcd(h_{1},\dots,h_{N})}$.

\end{frame}

\begin{frame}{Case 2 - Generalization}
    We want to find $f_{1}$ such that :
\begin{displaymath}
    \left\{
    \begin{array}{ll}
        f_{1} = x-g_{1} \pmod {y_{1}} \\
        f_{1} = x-g_{2} \pmod {y_{2}} \\
        \hspace{8mm} \vdots \\
        f_{1} = x-g_{N} \pmod{y_{N}}\\
    \end{array}
    \right.
\end{displaymath}
    Now, we want to find $f_{2}$ such that :
\begin{displaymath}
    \left\{
    \begin{array}{ll}
        f_{2} = f_{1} \pmod {\prod_{i=1}^{N} y_{i}} \\
        f_{2} = \prod^{N}_{i=1} (x-g_{i}) \pmod {\gcd(\{h_{i}\})} \\
    \end{array}
    \right.
\end{displaymath}
We compute $f_1$ and $f_2$ using CRT.

\begin{theorem}
    When $gcd(h_{1},\dots,h_{N})\neq1\ \forall i \in \{1,\dots,N\}$ and $gcd(h_{i},h_{i+1})=gcd(h_{i+1},h_{i+2})\ \forall i \in \{1,\dots,N-2\}$
    \begin{displaymath}
        I_{1} \cap I_{2} \cap \dots \cap I_{N} = \langle lcm(\{h_{i}\}), gcd(\{h_{i}\})f_{1},f_{2} \rangle
    \end{displaymath}
\end{theorem}
\end{frame}

\begin{frame}{References}
    \nocite{*}
    \printbibliography
\end{frame}

\end{document}